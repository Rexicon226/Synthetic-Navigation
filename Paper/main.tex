\documentclass{article}

% Packages
\usepackage{natbib}

\title{Synthetic Navigation: Machine Learning Autonomous Navigation Using Predicted and Ground-Truthed Maps}

\author{
    David Rubin \\
    Horizons CIS \\
    % \texttt{daviru007@example.com}
    \and
    Marcus E. Kauffman \\
    De La Salle High School \\
    % \texttt{marcus email}
}

\date{\today}

\begin{document}

\maketitle

\begin{abstract}
    TODO: abstract
\end{abstract}

\section{Rationale}
    Navigation over uncharted terrain has always required precise instruments
and careful data collection. Thanks to modern technology, such as satellite 
systems and GPS, no area is left completely undocumented \citep{deschamps-berger2020, kervyn2007, li1988}. 
Those who want to autonomously explore areas are left with two options: 
they can use an open-loop navigation system, relying solely on potentially 
inaccurate collected data, or they can use a closed-loop system which uses 
only sensor (ground-truthed) input, disregarding any preexisting data. 
This preexisting data, while it cannot be relied on, can be useful for 
navigation. 

    A pathfinding situation consists of a “predicted” map, which is available 
immediately and a “ground-truthed”, “observed” or “actual” map which is 
revealed to the algorithm or model as it explores the map. These two maps 
have a variation between them, the degree of which is referred to as the 
“edit distance” and expressed as a percentage, where 0\% is zero difference 
between the maps and 100\% means that the second map was generated separately 
from the first map. The combination of both maps is referred to as the 
“environment,” the shortest possible path between two points, the “start” 
and the “goal,” is the “best path” or “shortest path,” and the path that 
the algorithm takes is the “path” or “chosen path.” Due to the iterative 
nature of the D* algorithm and the machine learning model, the chosen path 
will most often not be the best path. 

    The most popular open-loop pathfinding algorithm is A* (pronounced “a-star”) 
uses node based weighted pathfinding and prioritizes pathfinding in directions 
that appear to be better. D* is an adaptation of A* for dynamic environments 
using an “incremental search strategy” when the final environment cannot be 
determined (X. Sun et al., 2010). 

    Many different techniques for machine learning 
(ML) exist (Amit Patel, 2022). For this application, reinforcement learning 
is best as it has shown success with pathfinding (Roy et al., 2017) and is 
able to make a choice between exploration and using past data which is necessary 
for using a combination of both maps. 

    This research will describe the 
feasibility and limits of the use of a ML model for navigation of real 
autonomous vehicles compared to traditional (A* and D*) pathfinding algorithms.

% Bibliography
\bibliographystyle{plainnat}
\bibliography{references}

\end{document}
